\documentclass{article}

\title{Hyperlinks in Latex}
\author{sachin rawat}
\date{May 2024}
          
\begin{document}

\maketitle    

% Introduction section with a label for reference}
\section{Introduction}
\label{sec:intro}
Welcome to this guide on using hyperlinks in LaTex. This document will cover various topics related to hyperlinking, including setting up options and referencing different elements.

\section{Setup Section}
\label{sec:setup}
Setting up a document with hyperref is straightforward. You can import `hyperref` with default settings or define custom options.

% Section referencing
\section{Hyperlink Options}
\label{sec:options}
Hyperref can be customized to suit your needs. Options such as `colorlinks`, `linkcolor` and `urlcolor` allow you to change the appearance of links.\\

In LaTex, you can hide or remove link formatting while preserving the functionality of the links. This can be useful if you want to maintain a consistent look in your document without visible link borders or color changes.

% Referencing a section 
\section{Referencing a Section}
\label{sec:section}
Referencing a section in LaTex means creating a link or cross-reference to another part of the document, specifically to a section with a defined heading (such as Introduction, Methods, Results, etc.). This allows you to refer to different parts of your document without repeating the information.

% Referencing an equation 
\section{Referencing an Equation}
\label{sec:equation}
This is the famous Einstein equation 

% Referencing a page 
\section{Referencing an Page}
\label{sec:page}
% Linking to a specific page
This document is generated with LaTex, a high-quality typesetting system. Hyperref allows you to navigate through various sections with ease. The previous information was found on page

% Link to an external web address
\section{External Links}
To find more information on LaTex, visit the LaTex Project website

% Cross linking sentences
\section{Cross-link Sentences}
You can also cross-link any sentences within your document.


\section{Einstein Equation}
\begin{equation}
E = mc^2
\label{eq:einstein}
\end{equation}


\end{document}
